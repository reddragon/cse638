\documentclass{article}
\usepackage{graphicx}
\usepackage{amssymb}
\usepackage{amsmath}
\usepackage{tikz}
\usepackage{url}
\usepackage{color}
\usepackage{savetrees}
\usepackage{listings}
\usetikzlibrary{shapes}


\linespread{1.5}
\setlength{\parindent}{0pt}
\setlength{\parskip}{1.9ex plus 0.5ex minus 0.2ex}

\title{CSE638 \\ Advanced Graduate Algorithms (and Data Structures) -- Homework-1}

\author{Dhruv Matani (108267786) \& Gaurav Menghani (108266803)}

\begin{document}
\maketitle

\clearpage

\tableofcontents

\clearpage

\section{Majority Problem}
Let us select a sample of $s = c\log{n}$ elements. In the worst-case, we can
have the array with just two types of elements, the majority element and the
non-majority element.

Now, after drawing the sample of $s$ elements, let the random variables
$x_{1}$ to $x_{s}$ define whether the $i$-th element was the actual majority
element or a non-majority element.

Let,

\[
  x_{i} = \left\{
  \begin{array}{l l}
    0 & \quad \text{if the element was a non-majority element}\\
    1 & \quad \text{if the element was a majority element}\\
  \end{array}
  \right.
\]

Let, $X = \displaystyle\sum\limits_{i=1}^s{x_{i}}$,
given that atleast 55\% of the elements in the array are the majority elements,
$E[X] \geq \dfrac{55}{100}s$. We pick a false majority when $X < 50$. Now,
let us find the probability of that happening.

Using Chernoff Bounds, we know,\\
$Pr(X < (1-\delta)\mu) < \left(\dfrac{e^{-\delta}}{(1-\delta)^{(1-\delta)}}\right)^\mu$.\\
\\
Now, $(1-\delta)\mu = \dfrac{50}{100}s$.\\
\\
$(1-\delta)\dfrac{55}{100}s = \dfrac{50}{100}s$.\\
\\
$\delta = \dfrac{1}{11}$.\\
\\
Now, what is $Pr($We choose a wrong majority element$)?$\\
$ = Pr(X < \dfrac{50}{100}s) = \left(\dfrac{e^{-\frac{1}{11}}}{(\frac{10}{11})^{\frac{10}{11}}}\right)^{\frac{55}{100}s}$\\
\\$ = Pr(X < \dfrac{50}{100}s) = \left(\dfrac{e^{-\frac{1}{20}}}{(\frac{10}{11})^{\frac{1}{2}}}\right)^s$\\
\\$ = Pr(X < \dfrac{50}{100}s) = \left(\dfrac{1}{1.00234^{s}}\right)$\\
\\$ = Pr(X < \dfrac{50}{100}s) = \left(\dfrac{1}{1.00234^{c\log{n}}}\right)$\\
\\$ = Pr(X < \dfrac{50}{100}s) = \left(\dfrac{1}{n^{c\log{1.00234}}}\right)$\\
\\
Hence, the probability of finding a fake majority is polynomially small for sufficiently large $c$.
\clearpage

\section{Expected number of bins with exactly one ball?}

\end{document}
