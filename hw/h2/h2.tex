\documentclass{article}
\usepackage{graphicx}
\usepackage{amssymb}
\usepackage{amsmath}
\usepackage{tikz}
\usepackage{url}
\usepackage{color}
\usepackage{savetrees}
\usepackage{listings}
\usetikzlibrary{shapes}


\linespread{1.5}
\setlength{\parindent}{0pt}
\setlength{\parskip}{1.9ex plus 0.5ex minus 0.2ex}

\title{CSE638 \\ Advanced Graduate Algorithms (and Data Structures) -- Homework-1}

\author{Dhruv Matani (108267786) \& Gaurav Menghani (108266803)}

\begin{document}
\maketitle

\clearpage

\tableofcontents

\clearpage

\section{Majority Problem}
Let us select a sample of $s = c\log{n}$ elements. In the worst-case, we can
have the array with just two types of elements, the majority element and the
non-majority element.

Now, after drawing the sample of $s$ elements, let the random variables
$x_{1}$ to $x_{s}$ define whether the $i$-th element was the actual majority
element or a non-majority element.

Let,

\[
  x_{i} = \left\{
  \begin{array}{l l}
    0 & \quad \text{if the element was a non-majority element}\\
    1 & \quad \text{if the element was a majority element}\\
  \end{array}
  \right.
\]

Let, $X = \displaystyle\sum\limits_{i=1}^s{x_{i}}$,
given that atleast 55\% of the elements in the array are the majority elements,
$E[X] \geq \dfrac{55}{100}s$. We pick a false majority when $X < 50$. Now,
let us find the probability of that happening.

Using Chernoff Bounds, we know,

$Pr(X < (1-\delta)\mu) < \left(\dfrac{e^{-\delta}}{(1-\delta)^{(1-\delta)}}\right)^\mu$.

Now, $(1-\delta)\mu = \dfrac{50}{100}s$.

$(1-\delta)\dfrac{55}{100}s = \dfrac{50}{100}s$.

$\delta = \dfrac{1}{11}$.

Now, what is $Pr($We choose a wrong majority element$)?$

$ = Pr(X < \dfrac{50}{100}s) = \left(\dfrac{e^{-\frac{1}{11}}}{(\frac{10}{11})^{\frac{10}{11}}}\right)^{\frac{55}{100}s}$

$ = Pr(X < \dfrac{50}{100}s) = \left(\dfrac{e^{-\frac{1}{20}}}{(\frac{10}{11})^{\frac{1}{2}}}\right)^s$

$ = Pr(X < \dfrac{50}{100}s) = \left(\dfrac{1}{1.00234^{s}}\right)$

$ = Pr(X < \dfrac{50}{100}s) = \left(\dfrac{1}{1.00234^{c\log{n}}}\right)$

$ = Pr(X < \dfrac{50}{100}s) = \left(\dfrac{1}{n^{c\log{1.00234}}}\right)$

Hence, the probability of finding a fake majority is polynomially small for sufficiently large $c$.

\clearpage

\section{Balls \& Bins again}

We use the concept of \textit{indicator random variables} to solve
this problem\footnote{Partial solution from
  \url{http://math.stackexchange.com/questions/28930/another-balls-and-bins-question}}.

\subsection{Excepted number of empty bins}

After $n$ throws of balls, let $Y$ be the number of bins that have no
balls in them. Let $X_i$ be the indicator variable so that $X_i$ is
$1$ if bin $i$ is \textbf{empty} and is $0$ otherwise. Hence, 

$Y = X_1 + X_2 + X_3 + \ldots{} + X_n$

By the linearity of expectation,

$E[Y] = E[X_1] + E[X_2] + E[X_3] + \ldots{} + E[X_n]$

However, for every $i \in [1\ldots{}n]$, $E[X_i] = 0 \times P(X_i = 0)
+ 1 \times P(X_i = 1)$.\\
$P(X_i = 1) = $ the probability that bin $i$ is empty\\
$\Rightarrow P(X_i = 1) = $ the probability that all $n$ balls went
into a bin other than bin $i$\\
$\Rightarrow P(X_i = 1) = (1 - \frac{1}{n})^n$\\
$\therefore E[X_i] = 1 \times (1 - \frac{1}{n})^n$\\
$\Rightarrow E[Y] = n \times (1 - \frac{1}{n})^n$\\
$\Rightarrow E[Y] = n \times \frac{1}{e}$\\
$\Rightarrow E[Y] = \frac{n}{e}$


\subsection{Expected number of bins with exactly one ball}

After $n$ throws of balls, let $Y$ be the number of bins that hold
exactly one ball. Let $X_i$ be the indicator variable so that $X_i$ is
$1$ if bin $i$ has \textbf{exactly one} ball and is $0$
otherwise. Hence,

$Y = X_1 + X_2 + X_3 + \ldots{} + X_n$

By the linearity of expectation,

$E[Y] = E[X_1] + E[X_2] + E[X_3] + \ldots{} + E[X_n]$

However, for every $i \in [1\ldots{}n]$, $E[X_i] = 0 \times P(X_i = 0)
+ 1 \times P(X_i = 1)$.\\
$P(X_i = 1) = $ the probability that bin $i$ has exactly one ball\\
$\Rightarrow P(X_i = 1) = $ number of ways of selecting a ball \textit{and} the probability that one ball went into
bin $i$ \textit{and} all other other $(n-1)$ balls went into a bin other than
bin $i$\\
$\Rightarrow P(X_i = 1) = \binom{n}{1} \times \frac{1}{n} \times (1 - \frac{1}{n})^{(n-1)}$\\
$\therefore E[X_i] = n \times \frac{1}{n} \times \frac{1}{e}$\\
$\therefore E[X_i] = \frac{1}{e}$\\
$\Rightarrow E[Y] = n \times \frac{1}{e}$\\
$\Rightarrow E[Y] = \frac{n}{e}$

\clearpage

\section {Problem 2 Using W.H.P. Analysis}

\subsection{Negative Correlation}

\subsection{$\theta(n)$ empty and $\theta(n)$ full bins, with high probability}
In Question 2, we proved that $E[$\# of empty bins$] = \dfrac{n}{e}$. \\
Therefore,
$E[$\# of non-empty bins$] = \left(n-\dfrac{n}{e}\right) = \left(\dfrac{ne-n}{e}\right)$.\\
\\
Now, let us prove that the probability of having $\leq \dfrac{n}{10e} $ empty bins
is polynomially small.\\
\\
Using Chernoff Bounds, we know,
\\
$Pr(X < (1-\delta)\mu) < \left(\dfrac{e^{-\delta}}{(1-\delta)^{(1-\delta)}}\right)^\mu$.
\\
Now, $(1-\delta)\mu = \dfrac{n}{10e}$\\
\\
$(1-\delta)\dfrac{n}{e} = \dfrac{n}{10e}$\\
\\
$\delta = \dfrac{9}{10}$\\
\\
$Pr\left(X < \dfrac{n}{10e}\right) < \left(\dfrac{e^{-\frac{9}{10}}}{(1-\frac{9}{10})^{(1-\frac{9}{10})}}\right)^{\frac{n}{e}}$\\
\\
$Pr\left(X < \dfrac{n}{10e}\right) < \left(\dfrac{e^{-\frac{9}{10}}}{(\frac{1}{10})^{(\frac{1}{10})}}\right)^{\frac{n}{e}}$\\
\\
$Pr\left(X < \dfrac{n}{10e}\right) < \left(\dfrac{1}{e^{\frac{9}{10}}(\frac{1}{10})^{(\frac{1}{10})}}\right)^{\frac{n}{e}}$\\
$Pr\left(X < \dfrac{n}{10e}\right) < \left(\dfrac{1}{1.2793^{n}}\right)$\\
\\
Thus, the probability of having $\leq \dfrac{n}{10e} $ empty bins
is exponentially small.\\
\\
Now, let us prove that the probability of having $\leq \dfrac{ne-n}{10e} $ full bins
is polynomially small.\\
\\
Using Chernoff Bounds, we know,
\\
$Pr(X < (1-\delta)\mu) < \left(\dfrac{e^{-\delta}}{(1-\delta)^{(1-\delta)}}\right)^\mu$.
\\
Now, $(1-\delta)\mu = \dfrac{ne-n}{10e}$\\
\\
$(1-\delta)\dfrac{ne-n}{e} = \dfrac{ne-n}{10e}$\\
\\
$\delta = \dfrac{9}{10}$\\
\\
$Pr\left(X < \dfrac{ne-n}{10e}\right) < \left(\dfrac{e^{-\frac{9}{10}}}{(1-\frac{9}{10})^{(1-\frac{9}{10})}}\right)^{\frac{ne-n}{e}}$\\
\\
$Pr\left(X < \dfrac{ne-n}{10e}\right) < \left(\dfrac{e^{-\frac{9}{10}}}{(\frac{1}{10})^{(\frac{1}{10})}}\right)^{\frac{ne-n}{e}}$\\
\\
$Pr\left(X < \dfrac{ne-n}{10e}\right) < \left(\dfrac{1}{e^{\frac{9}{10}}(\frac{1}{10})^{(\frac{1}{10})}}\right)^{\frac{ne-n}{e}}$\\
$Pr\left(X < \dfrac{ne-n}{10e}\right) < \left(\dfrac{1}{1.527^{n}}\right)$\\
\\
Therefore, the probability of having $\leq \dfrac{ne-n}{10e} $ full bins, is exponentially small.\\
\\
Hence, we can say that w.h.p, there are $\theta(n)$ empty and $\theta(n)$ full bins.

\subsection{Part (b) without Chernoff Bounds}
\clearpage

\section {Problem 4}

\clearpage

\section {Problem 5}

\clearpage
\end{document}
