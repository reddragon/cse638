\documentclass{article}
\usepackage{graphicx}
\usepackage{amssymb}
\usepackage{amsmath}
\usepackage{tikz}
\usepackage{url}
\usepackage{color}
\usepackage{savetrees}
\usepackage{listings}
\usetikzlibrary{shapes}


\linespread{1.4}
\setlength{\parindent}{0pt}
\setlength{\parskip}{1.9ex plus 0.5ex minus 0.2ex}

\title{A Simple Solution to the Level Ancestor Problem}

\author{Gaurav Menghani \& Dhruv Matani}

\begin{document}
\maketitle


\begin{abstract}

In this paper, we present a simple $O\mathrm{<}n, \log{n}\mathrm{>}$
approach to solving the \textit{Level Ancestor Problem}.

\end{abstract}

\section{Fast Binary Search}
Given a sorted array, the standard binary search takes $O(\log{n})$ steps. 
We can pre-process the sorted array in $O((\log{\log{n}})^3)$ (FIXME) time 
such that we can perform this search in $O((\log{\log{n}})^3)$ (FIXME) time.

Given an array of $n$ sorted elements, we subdivide the array into chunks of 
size $\log{n}$ each. There would be $ \frac{n}{ \log{n} }$ chunks in all. For each 
chunk, we choose a representative element, which is the first element in each 
chunk, and store these elements, in order, in a new array of size $\frac{n}{\log{n}}$. 
Now, if we need to search for an element, we can first search amongst the 
array of representative elements at a cost of $O(\log{\frac{n}{\log{n}}})$, 
if we do not find the element in this array, we can use the largest element 
in the represntative element array which is smaller than our target element to 
index into the correct chunk of the original array. Now, we only need to search 
in the chunk we indexed into, and this would take $O(\log{\log{n}})$ time since 
the chunk is of size $\log{n}$.

We can follow this approach recursively, and the analysis below shows that this
preprocessing takes $O((\log{\log{n}})^3)$ (FIXME) time, and a query takes
$O((\log{\log{n}})^3)$ (FIXME) time. The space requirement of the approach is 
$O(\log{n})$?? (FIXME).

\section{Level Ancestor Problem}
Do a Euler Tour of the tree, and label the nodes in increasing order of their
first appearance in the Euler Tour. We maintain a vector for each level in the 
tree, and as and when we see nodes for the first time, we insert them into their
labels into the respective vectors (as per the level of the node). It is easy
to see that the labels in each vector would be in increasing order.

Now, if we wish to answer the $l$-th Level Ancestor query for a node labelled $n$,
we can index into the vector for the $l$-th level, and binary search for the 
greatest label in the vector, which is smaller than $n$. The reason why this 
works is that the ancestor at level $l$ would have been visited before the node 
labelled $n$, and hence have a smaller label. Any other node on level $l$ would have
either been visited before the correct ancestor, or after the node $n$. Using
Binary Search, we can thus answer the Level Ancestor query in $O(\log{n})$ time, 
with $O(n)$ extra space but without any preprocessing. 

Using the fast binary search we described earlier, we can improve the running 
time to $O((\log{\log{n}})^3)$ (FIXME) time, without increasing the extra space
requirement asymptotically.

\end{document}
